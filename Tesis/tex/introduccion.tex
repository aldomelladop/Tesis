\chapter{Introducción}

\section{Antecedentes históricos}
El ser humano, en la década del 60, ante la incipiente necesidad de saber su posición en el planeta desarrolló un sistema de posicionamiento  llamado OMEGA y posteriormente otro llamado TRANSIT o \ac{NAVSAT}, que  fue resultado del trabajo conjunto de la NASA y el departamento de defensa de los Estados Unidos. Años después este acabó siendo reemplazado debido a la falta de precisión que este tenía, y que alcanzaba un error de hasta 250 metros.

Su sucesor apareció en la década del 70 bajo el nombre de \ac{GPS} y su precisión permitía posicionar un objeto con un error de menos de 5 metros. Esto a través del cálculo del tiempo que tarda en llegar la señal al receptor, es decir, el efecto Doppler.

Funciona actualmente con un mínimo de 24 satélites en órbita sobre la tierra cuyas trayectorias sincronizadas le permiten mapear completamente el planeta y entregar posicionamiento casi exacto a dispositivos móviles y vehículos.

\section{Definición del problema}
Las condiciones bajo las cuales es posible para la señal propagarse no se cumplen en todos los ambientes. Existen lugares en que a diferencia de lo que sucede en el exterior, donde la señal se refleja haciendo posible la triangulación de la posición, esta se absorbe parcial o completamente y principalmente corresponden a espacios interiores, tales como una bodega, un centro comercial o una oficina, haciendo que posicionarse dentro de estos espacios sea imposible a través del GPS, es por esto que a fin de brindar nuevas experiencias a usuarios a través del posicionamiento dentro de estos espacios se han desarrollado soluciones utilizando la banda de los 2.4 [GHz] que es la utilizada por, entre otros tecnologías, el Wi-fi. 

Esta ha sido ampliamente estudiada debido a la alta penetración comercial que ha alcanzado precisamente en estos espacios donde el GPS no da cobertura, sin embargo, dentro de los distintos enfoques en que se han abordado los estudios se tiene que la medición de los niveles de potencia radiada desde los \ac{AP}, a través de los cuales se realiza la triangulación de la posición, se ven altamente afectados por la variación del escenario caracterizado. Este tipo de variaciones pueden ser inducidas por la presencia de personas u objetos que reflejen o absorban la señal.

Es por esto que el método a través del cual se de solución al problema del posicionamiento indoor debe poder compensar estas variaciones de potencia inducidas y estimar correctamente la posición de objetivo deseado identificable a través de su dirección de \ac{MAC}.

\section{Estado del arte}

\section{Hipótesis de trabajo}
\begin{center}
\textit{''Es posible posicionar un dispositivo, identificable a través de su MAC, en un espacio interior utilizando una \ac{RNN} para compensar las pérdidas en los niveles de potencia de los \ac{AP} usados en la triangulación''}
\end{center}

\section{Objetivos}
A continuación se señalan los objetivos que apuntan a resolver el problema presentado y a probar la hipótesis de trabajo.

\section{Objetivo general}
\section{Objetivo específico}
\section{Alcances y limitaciones}
\section{Metodología}