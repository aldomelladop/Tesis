\chapter{Conclusiones}

% ***********************************************************************************
\section{Sumario}
Dentro del desarrollo del estudio, se contemplaron diversos enfoques, se ponderó la posiblidad de implementar en la memoria de título un sistema de posicionamiento indoor basándose en algoritmos de estimación de posición geométricos y otros, en parámetros más complejos, no obstante, la poca practicidad y adaptabilidad de estos modelos al ambiente donde se podría implementar, supone una desventaja innegable.

% ***********************************************************************************
\section{Conclusiones}
Por todo lo revisado, estudiado y expuesto, se plantea la posibilidad de que, como alternativa a los algoritmos geométricos o basados en potencia, se consideren alternativas más efectivas, de menor coste computacional y que no dependan de parámetros relacionados con el ambiente, dado que en ambientes indoor o en sectores muy concurridos hay métricas que varían considerablemente, al punto de obligar a realizar mediciones correctivas.

Además, se tiene que la efectividad y precisión del método que recoge las solicitudes IRDP, es considerablemente superior al no depender de otros parámetros salvo la llegada de los paquetes al router.

Se dio cumplimiento a los objetivos planteados inicialmente, mediante la investigación de las alternativas, que fueron abordadas a fondo a través de literatura. Se pudo llegar a una comprensión justificada de las ventajas y desventajas de los métodos, pudiendo así, escoger con base en distintos estudios teóricos y empíricos, cuál es el método que convendría desarrollar en detalle en la memoria de título.

En suma, se destaca la importancia de llegar a desarrollar un sistema de posicionamiento indoor que cuente con una precisión que no sea directamente proporcional a la capacidad de cómputo, tanto si se desea resolver necesidades humanas tales como, hallar la mejor ruta de acceso para una persona con capacidades diferentes, a través de una aplicación móvil que desde el seguimiento de su interfaz vaya dándole instrucciones. Como si se desea implementar un sistema que recoja las solicitudes de los clientes que desean conectarse y con ello, hacer un seguimiento de las posiciones, a fin de que este, permita generar reportes e informes estadísticos que relacionen patrones de conducta y también de consumo, que favorezcan a un mejoramiento en las estrategias de venta y en la experiencia de compra del cliente.

% ***********************************************************************************
\section{Trabajo Futuro}
Se listan las posibles líneas de investigación que se deducen directamente de la presente obra.

\begin{enumerate}
\item {Implementación de un sistema que realice el seguimiento de las solicitudes IRDP donde pueda visualizarse el movimiento de las interfaces de red}
\item{Realizar un estudio que contraste la precisión y eficacia con la que los métodos geométricos y el de las solicitudes IRDP realizan la estimación de posición.}
\item{Estudiar algoritmos de posicionamiento basado en redes neuronales.}
\item{Estudiar la precisión de las mediciones obtenidas por el uso del sistema presentado en \cite{8}}
\end{enumerate}

Se quiere que al menos existan unas 3 a 5 ideas en las que se pueda seguir investigando.
% ***********************************************************************************
%\section{Publicación}
%A veces, los trabajos dan pie a una publicación en conferencia o revista. Es importante mencionarlo en esta sección.